\section{Discussions}

M1. Scalability: The MF should be scalable i.e. it can cope with a large number of monitoring data col- lectors. This requirement is very important in Cloud Computing scenarios due to a large number of param- eters to be monitored for potentially large amount of services and elements of Cloud tiers that may grow elastically. M2. Heterogeneous data: The MF should consider a heterogeneous group of metrics. The MF must al- low the collection of service level runtimemonitoring data, virtual IT-infrastructure monitoring data (e.g., VMlevel runtimemonitoring), and fine-grained phys- ical IT-infrastructure monitoring data (e.g., network links, computing and storage resources). M3. Polling Interval: The data collection mecha- nism must allow the dynamic customization of the polling interval. Dynamic nature of virtual platforms demands gathering of data in a sufficiently frequent manner, meaning that nodes should be monitored continuously. Naturally, smaller polling intervals in- troduce significant processing overhead inside nodes themselves. However, long polling intervals do not provide a clear picture of the monitored components. Therefore, an optimal trade-off between polling inter- val and processing overhead is required. M4. Relationship: In the abovementioned scenario, clusters of VMs and Physical Machines (PM) serve many kinds of applications, so there is a hierarchi- cal relationship between applications, VMs and PMs.

There is also a possibility of migration of VMs and applications from one node to another, so relation- ships can be changed dynamically. Themetric’s value must be tagged to showthat they belong to a particular instance (e.g., application), and what is their relation to other instances (e.g., VMand PM). M5. Data Repository: The MF requires a data repository where raw monitoring data needs to be stored after collection. The original data set must be stored without down-sampling for auditing pur- poses. The stored, raw monitoring data can be re- trieved by consumers to perform QoS fault diagno- sis, SLA validation, plot rendering, and as an input for fine grained resource management. The database must be distributed in order to avoid single point of failure. Moreover, it must be scalable, and allow to store thousands of metrics and potentially billions of data points. M6. Non-Intrusive: TheMFmust be able to retrieve data non-intrusively froma variety of sources (forVM via libvirt API, for a host via cgroups, for the net- work via SNMP, for Java applications via JMX, etc.). Collection mechanism should easily be extensible by adding more plugins. M7. Interface: The MF should provide a REST interface that allows access to the current monitor- ing data in a uniform and easy way, by abstracting the complexity of underlying monitoring systems. A standard unified interface for common management and monitoring tasks can make different virtualiza- tion technologies and Cloud providers interoperable. A REST interface is a good choice due to ease of im- plementation, low overhead and good scalability due to its session-less architecture.