\section{Introduction}\label{sec:Introduction}
%
%1 - Challenges in understanding performance on cloud environment.
Majority of the applications nowadays are moving to cloud-based environment, with the market size grows to $\$246.8B$ in 2017, according to Forbes\footnote{http://bit.ly/forbescloudapps2017}. Cloud computing provides the users with the possibility of using different devices to use (access) the cloud-based services seamlessly \cite{Mell2011}. It, however, brings more challenges to application developers and architects to understand how the applications work and how their customer satisfy with the provided services. To have an example, let us recall the scenario of the cloud-based picture. Cloud providers deliver the same service to different customers, who share physical and/or virtual resources transparently, this concept is referred as Multi-tenancy \cite{Kabbedijk2015}.  A cloud-based application lets customers share the same hardware resources, by offering them one shared application and database instance, while allowing them to configure the application to fit their needs as if it runs on a dedicated environment \cite{Zaidman2010}. End-users or consumers interact with the cloud applications through various interfaces (those being web browsers, mobile applications, and command-line interfaces). Cloud services allows different users to share computing and virtual resources transparently, meanwhile guaranteeing substantial cost savings \cite{Bezemer2010}. The Infrastructure as a Service (IaaS) model offers computer - physical or virtual machines - and other resources, such as raw block storage, file or object storage, virtual local area networks (VLANs), IP addresses, and firewalls. In the Software as a Service (SaaS) model, applications deployed on cloud infrastructure and are provided to end-users as services over the internet. The more resources are managed by cloud providers, the more resources are shared by multiple different users. Extraction of usage data of the features provided by the cloud applications could help software developers and architects to make an informed decision for the development/improvement of functionalities of the system according to end-user usage patterns \cite{Pachidi2014}. 

%2 - Why do we need analytical solutions
Analytical solutions refer to the use of various analysis techniques and methods such as data mining, machine learning, reasoning, and etc. to extract useful knowledge and insights from large data set. For example, enterprises can use analysis techniques to understand customers' behaviour and predict which customers are least likely to quit, or most likely to respond to a particular feature of the application based on their profiles, memberships they subscribe to, and their generated content (comments, clicks, etc.). Developers can understand if some functions do not work properly via the usage data generated by the users. User interests can be modelled by extracting browsing behavior when accessing web application \cite{Gasparetti2016}. Such analytical solutions are considered as increasingly important tools for modern enterprise to get an informational advantage, and have evolved from a matter of choice to a necessity in some crowded and competitive business environments. Applying analytical solutions, thus, is a key to discover insights from the applications' usage. Traditionally, in order to understand how customers use a service, enterprises have to make a huge data collection via surveying and feedback from testers. With analytical solutions, such insights can be discovered via advance machine learning and data mining techniques from the usage data. 


Usage analytics aims to obtain insightful and actionable information for data-driven tasks, around applications and services. Insightful information is information that conveys meaningful and useful understanding or knowledge towards providing the target service or user satisfaction to that service \cite{Zhang2011}. Typically insightful information is not easily attainable by directly investigating the raw data without aid of analytical solutions. Actionable information is information upon which software practitioners can come up with concrete solutions (better than existing solutions if any) towards completing the target task. Developing a usage analytics project typically goes through iterations of the life cycle of four phases: task definition, data preparation, analytic-technology development, and deployment and feedback gathering. Task definition is to define the target task to be assisted by software analytics. Data preparation is to collect data to be analyzed. Analytic-technology development is to develop problem formulation, algorithms, and systems to explore, understand, and get insights from the data. Deployment and feedback gathering involves two typical scenarios. One is that, as researchers, we have obtained some insightful information from the data and we would like to ask domain experts to review and verify. The other is that we ask domain experts to use the analytic tools that we have developed to obtain insights by themselves. Most of the times it is the second scenario that we want to enable. Among various analytic technologies, machine learning is a well recognized technology for learning hidden patterns or predictive models from data. It plays an important role in the software analytics domain. 

In this paper, we argue that when applying analytic technologies in practice of software analytics, one should:

- incorporate a broad spectrum of domain knowledge and expertise, e.g., management, machine learning, large-scale data processing and computing, and information visualization;
- investigate how practitioners take actions on the produced insightful and actionable information, and provide effective support for such information-based action taking.

To summarise, by using usage analytics, we are aiming at proposing powerful tools to address these challenges: 
%3 - What we can do with usage data analytics?
\begin{enumerate}
\item Resource provision
\item Problem diagnosis
\item User satisfaction
\item User behavior
\end{enumerate}

In this study, we summarise the big picture of how analytical solutions can be applied in cloud-based environments and answer some basis questions: 
\begin{itemize}
	\item ``What are the usage data?"
	\item ``What kind of insights can we get from the extracted data?"
	\item ``What kind of analytical techniques should be applied?"
\end{itemize}