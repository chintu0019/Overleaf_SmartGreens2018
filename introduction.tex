\section{Introduction}\label{sec:Introduction}
%
%1 - Challenges in understanding performance on cloud environment.
Major of the applications nowadays are moving to cloud-based environment, with the market size grows to $\$246.8B$ in 2017, according to Forbes\footnote{https://www.forbes.com/sites/louiscolumbus/2017/04/29/roundup-of-cloud-computing-forecasts-2017}. Users are using cloud-based services seamlessly, without recognising that they are not a desktop-based application. It, however, brings more challenges to people behind the systems to understand how the applications work and how their customer satisfy with the provided services. To have an example, let us recall the scenario of the cloud-based picture. Cloud providers deliver the same service to different customers, which share physical and/or virtual resources transparently. A cloud-based application lets customers share the same hardware resources, by offering them one shared application and database instance, while allowing them to configure the application to fit their needs as if it runs on a dedicated environment []. End-users or consumers interact with the cloud applications through various interfaces (those being web browsers, mobile applications, and command-line interfaces). Cloud services allows different users to share computing and virtual resources transparently, meanwhile guaranteeing substantial cost savings []. The Infrastructure as a Service (IaaS) model offers computer - physical or virtual machines - and other resources, such as raw block storage, file or object storage, virtual local area networks (VLANs), IP addresses, and firewalls. In the Software as a Service (SaaS) model, users can access applications and data. The more resources are managed by cloud providers, the more resources are shared by multiple different users. Extraction of usage data of the features provided by the cloud applications could help developers and architects to make an informed decision for the development/improvement of functionalities of the system according to end-user usage patterns []. 

%2 - Why do we need analytical solutions
Analytical solutions refer to the use of various analysis techniques (data mining, machine learning, reasoning, and etc.) to extract useful knowledge and insights from large data set. For example, enterprises can use analysis techniques to understand customers and predict which customers are least likely to quit, or most likely to respond to a particular offer based on their profiles, memberships they subscribe to, and their generated content (comments, clicks, etc.). Developers can understand if some functions do not work properly via the usage data generated by the users. Moreover, the resulting predictions can be used to generate lists of target customers or cases of interest, as input for strategic planning, or can be integrated into the context of other enterprise applications. Such analytical solutions are considered as increasingly important tools for modern enterprise to get an informational advantage, and have evolved from a matter of choice to a necessity in some crowded and competitive business landscapes. Applying analytical solutions, thus, is a key to discover insights from the applications. Traditionally, in order to understand how customers using a service, enterprises have to make a huge data collection via surveying and feedback from testers. With analytical solutions, such insights can be discovered via advance machine learning and data mining techniques from the usage data. By using usage analytics, we are aiming at proposing powerful tools to address these challenges: 

%3 - What we can do with usage data analytics?
\begin{enumerate}
\item Resource provision
\item Problem diagnosis
\item User satisfaction
\item User behavior
	
\end{enumerate}

In this study, we summarise the big picture of how analytical solutions can be applied in cloud-based environments and answer some basis questions: 
\begin{itemize}
	\item ``What are the usage data?"
	\item ``What kind of insights can we get from the extracted data?"
	\item ``What kind of analytical techniques should be applied?"
\end{itemize}