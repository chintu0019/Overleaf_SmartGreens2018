\section{Realising the Potential}
All these potential held by --- comes also with challenges and opportunities for researchers.
It is important that the multimedia community helps to address the challenges in this emerging and important field. 
We cannot easily apply our existing methods on this type of data and hope for access. Therefore we need specific
approaches addressing the specific challenges.
As a first way-point for researchers we are proposing different research topics and research questions.
\begin{enumerate}
	\item How can we identify and extract important information? Deciding what should be extracted from the data that is important for the users is a nontrivial task. Going beyond standard analysis like detecting visual objects or text will be important, forcing researchers to think creatively and go beyond simple analysis. Many research questions arises, such as, how to combine information from sensors with videos or images, or how to efficiently process the data. Also an important part here is to explore how context can be taken into account to improve the quality of the analysis.
	
	\item How can we present the results to the users? 
	Reporting the results to the users is one of the most important parts of the analysis of this data. Nevertheless, this is not trivial since the amount of data and information that can be extracted is huge. It will be important to research novel interfaces that allow users to browse efficiently. Apart from that user feedback loops and recommender systems that can help the user to decide which information is important to them. Finally, generating summaries and automatic reports will be another topic that is important for this data since there will be a need from the user side for such summaries with respect to, for example, weekly report from sensors such as Fitbit smartwatches.
	
	\item How can and which technologies are need to make it easy to share, annotate and archive data? 
	While services similar to YouTube or Flickr might be interesting for users, it also comes with a privacy problem. Users might want to upload and share there data but in a more controlled way. New ways of secure sharing but also tools for annotating, redaction, and archiving large amounts of data will be needed. With this also comes the need for re-finding certain events which bring new challenges for search engines.  
	
	\item How can information and data be processed efficiently? Systems that have to process a huge amount of data in a complex way have to be efficient to make them useful to the users. This comes with challenges for multimedia systems researchers in terms of how to parallelize and process data efficiently in a reasonable amount of time.
\end{enumerate}

The potential for personal life archives is enormous. We do acknowledge that there are  challenges to be overcome, such as privacy concerns, data storage, security of data, and the development of a new generation of search and organisation tools, but we believe that these will be overcome and that we are on the cusp of a positive turning point for society; the era of the quantified individual who knows more about the self than ever before, has more knowledge to improve the quality of their own life and can share life events and experiences in rich detail with friends and contacts.